\chapter{概要}
近年, 機械学習分野の研究は非常に盛んである. その中で特に大規模言語モデル(LLM)
は, 人の仕事を代替できるなど(要検討) の理由から注目を集めている. 他方, 日本における
教育の変革が始まっている. 2020 年以降, 学習指導要領の改定が実施され, 多くの学校が
対応に追われている?. その中核に”主体的な学習”があり, 学生が自身から学ぶ姿勢を身に
着けることで, 激動の時代を生き抜ける力が付くという. その習得に貢献する方法として,
文部科学省はアクティブラーニングを検討, 議論していた. アクティブラーニングについ
て, 特に複数人で課題に取り組むグループワークは, 複数の研究からその有効性を示唆さ
れているが(???), 一方で課題も多くある(???). 我々はこの課題の内, 教師からの評価や生
徒の参加意欲に着目し,LLM によるフィードバックでそれらの改善を図るアプリケーショ
ン,???を開発した. 開発段階では, グループワーク中のディベート?を想定した評価実験を
行い,2 名の参加者から構想に関して肯定的な意見をいただいた. しかし, フィードバック
の詳細やUI については課題が残った. 今後はそれらを改善し, より洗練されたシステムを
目指す.