前提として,アプリケーション(以下アプリ)は中学校,高等学校のグループディスカッションで用いられることを想定しており,現時点では仮に5グループが存在すると考え,作成した.

本アプリは機能として,議事録の作成と評価を行うが,そのために前者では録音や音声認識を,後者でLLMの利用を行う.
また,教師用の機能としてユーザの登録がある.

アプリの流れは以下の通り:
1. Webでアプリを開く
2. 必要事項を入力し録音開始
3. 録音を終了し,その音声をGoogle2Speech APIに渡す
4. 音声認識の結果をINIAD AI MOPを活用してGPTに渡し,誤字脱字の修正や評価を実行


また,アプリは以下のページを持つ:
・生徒用
・教師用
・登録

生徒用ページは議事録の作成と評価を行う.
学生が1から5の一つをチーム番号として選択し,会議の日付とタイトルを入力後,録音ボタン押すと録音が開始する.

録音は学生がアプリを使っている端末のマイクで行うことを想定している.
現在中学校・高等学校では授業用の端末配布が進んでおり[参考文献],文科省の見立てでは20xx年にnn%を超えるとされている.
このことから,各チームは独自の端末を用いて録音が可能であると考えられる.

精度については実験で,周囲で音がする環境でPC備え付けのマイクを使って確認したところ,距離や位置関係は不明だが,周囲の話し声の影響を受けない可能性が示唆された.
