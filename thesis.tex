\documentclass[bachelor]{INIAD}%卒論用 
\addtolength{\footskip}{8mm}
\bibliographystyle{jplain} 
%\usepackage[dviout]{graphicx}
\usepackage[dvipdfmx]{graphicx}
\usepackage{bm}
\usepackage{amsmath}

%\usepackage{geometry}
%\geometry{left=30mm,right=30mm,top=35mm,bottom=30mm}

%\documentclass[oneside]{suribt}% 本文が * ページ以下のときに (掲示に注意)
\title{象の卵生についての研究}
%\titlewidth{}% タイトル幅 (指定するときは単位つきで)
\author{竹本 志恩}
\eauthor{Shion Takemoto}% Copyright 表示で使われる
\studentid{1F10220127}
%\supervisor{赤羽台 花子}% 1つの引数をとる (役職まで含めて書く)
%\supervisor{指導教員名 役職 \and 指導教員名 役職}% 複数教員の場合,\and でつなげる
\handin{2025}{1}% 提出月. 2 つ (年, 月) 引数をとる
%\keywords{キーワード1, キーワード2} % 概要の下に表示される
\renewcommand{\baselinestretch}{1.25}
\setcounter{tocdepth}{2}

\begin{document}
\mojiparline{40}
\maketitle%%%%%%%%%%%%%%%%%%% タイトル %%%%

\frontmatter% ここから前文

%\etitle{Title in English}

%\begin{eabstract}%%%%%%%%%%%%% 概要 %%%%%%%%
% 300 words abstract in English should be written here. 
%\end{eabstract}

\begin{abstract}%%%%%%%%%%%%% 概要 %%%%%%%%

\end{abstract}

%%%%%%%%%%%%% 目次 %%%%%%%%
{\makeatletter
\let\ps@jpl@in\ps@empty
\makeatother
\pagestyle{empty}
\tableofcontents
\clearpage}

\mainmatter% ここから本文 %%% 本文 %%%%%%%%

\include{01_intro.tex}     % はじめに
\include{02_related.tex}   % 関連研究
前提として,アプリケーション(以下アプリ)は中学校,高等学校のグループディスカッションで用いられることを想定しており,現時点では仮に5グループが存在すると考え,作成した.

本アプリは機能として,議事録の作成と評価を行うが,そのために前者では録音や音声認識を,後者でLLMの利用を行う.
また,教師用の機能としてユーザの登録がある.

アプリの流れは以下の通り:
1. Webでアプリを開く
2. 必要事項を入力し録音開始
3. 録音を終了し,その音声をGoogle2Speech APIに渡す
4. 音声認識の結果をINIAD AI MOPを活用してGPTに渡し,誤字脱字の修正や評価を実行


また,アプリは以下のページを持つ:
・生徒用
・教師用
・登録

生徒用ページは議事録の作成と評価を行う.
学生が1から5の一つをチーム番号として選択し,会議の日付とタイトルを入力後,録音ボタン押すと録音が開始する.

録音は学生がアプリを使っている端末のマイクで行うことを想定している.
現在中学校・高等学校では授業用の端末配布が進んでおり[参考文献],文科省の見立てでは20xx年にnn%を超えるとされている.
このことから,各チームは独自の端末を用いて録音が可能であると考えられる.

精度については実験で,周囲で音がする環境でPC備え付けのマイクを使って確認したところ,距離や位置関係は不明だが,周囲の話し声の影響を受けない可能性が示唆された.
  % 提案手法
開発 最初の発表まで
- 個人サーベイ テーマ決定
 - 全員教育関連でサーベイしてた
 - 経緯は議事録を参照
- チームサーベイ アプリ内容の議論
 - 教育の問題と既存の解決方法,生成AI活用事例を調査
 - 候補は多数出たが議事録の作成と評価に決定
 - 経緯は議事録へ
- 役割分担: 僕は機能の開発のはず
- スケジュール作成: そこまで無理のないスケジュールだったが,思ったよりAPIの選定やAmivoiceのAPIの利用に手間取り,機能面の開発が遅れた


夏季休暇開始から中間まで
- 毎週水曜前後にミーティング
 - 殿村君の提案 これはかなり良かった
 - ここを目標に小さな成果を積み上げられた
- 僕: 録音,認識などの機能を実装
 - 録音
  - 議事録を音声認識で生成するべく,音声の入手から実装
  - 
 - 
ajaxのエラー解決がその後も大変だった


中間発表後から年末まで
- 評価機能を実装
- 評価表示機能を実装
- エラーの解決


\include{05_review.tex}


% 以降、実装や評価、結論などの章を適切に配置してください

\backmatter% ここから後付
\chapter{謝辞}
           % 謝辞

\bibliography{thesis.bib}  % 参考文献

\appendix% ここから付録 %%%%% 付録 %%%%%%%
\chapter{評価に利用したコード例}
      % 付録

\end{document}
